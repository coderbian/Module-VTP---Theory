\begin{figure}[H]
    \centering
    \begin{tikzpicture}
        \begin{axis}[
            axis lines=middle,
            xtick={1,3,5},
            ytick={},
            xlabel={},
            ylabel={},
            xmin=-0.5, xmax=6,
            ymin=-2, ymax=5,
            width=10cm,
            height=7cm,
        ]
        
        % Nhánh 1: x < 3
        \addplot[domain=-0.5:3, samples=100, thick, violet!50!black, smooth] {-(x-1.5)^2 + 2.3};
        % Lỗ hổng tại x=1
        \node[circle, draw, fill=white, inner sep=1.5pt] at (axis cs:1, 2.05) {};
        
        % Nhánh 2: x > 3
        \addplot[domain=3:6, samples=100, thick, red!70!black, smooth] {0.3*(5 - x)^3 + 1.5};
        % Lỗ hổng tại x=5
        \node[circle, draw, fill=white, inner sep=1.5pt] at (axis cs:5, 1.5) {};
        
        % Điểm gián đoạn nhảy tại x=3
        \node[circle, draw, fill=white, inner sep=1.5pt] at (axis cs:3, 0.05) {};
        \node[circle, fill, inner sep=1.5pt] at (axis cs:3, 4) {};
        
        % Điểm rời rạc tại x=5
        \node[circle, fill, inner sep=1.5pt] at (axis cs:5, 4.5) {};

        % Điểm rời rạc tại x=2.5
        \node[circle, fill, inner sep=1.5pt] at (axis cs:2.5, -1.5) {};
        
        \end{axis}
    \end{tikzpicture}
    \caption{Minh họa đồ thị của một hàm số không liên tục (gián đoạn).}
    \label{fig:discontinous_func}
\end{figure}