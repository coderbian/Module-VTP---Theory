% --- Bắt đầu hình minh họa ánh xạ ngược và hợp ---
\begin{figure}[!h]
    \centering
    \begin{tikzpicture}[
        scale=0.9, transform shape,
        set/.style={ellipse, draw, minimum width=2.5cm, minimum height=3cm},
        element/.style={circle, fill, inner sep=1.5pt},
        arrow_f/.style={->, thick, bend left=25, blue!70!black},
        arrow_finv/.style={->, thick, bend left=25, red!70!black}
    ]
    % --- Hình 1: Ánh xạ ngược ---
    \begin{scope}[xshift=2cm, yshift=0cm]
        % Tập hợp
        \node[set] (X) at (0,0) {};
        \node[set] (Y) at (4,0) {};
        \node at (0, 1.8) {$X$};
        \node at (4, 1.8) {$Y$};
        
        % Phần tử
        \node[element, label=left:$x$] (x) at (0, 0) {};
        \node[element, label=right:$y$] (y) at (4, 0) {};
        
        % Mũi tên ánh xạ
        \draw[arrow_f] (x) to node[above, sloped] {$f$} (y);
        \draw[arrow_finv] (y) to node[below, sloped] {$f^{-1}$} (x);
        
        % Chú thích
        \node at (2, -2.2) {Ánh xạ ngược};
    \end{scope}
    
    % --- Hình 2: Ánh xạ hợp ---
    \begin{scope}[yshift=-5cm]
        % Tập hợp
        \node[set] (X) at (0,0) {};
        \node[set] (Y) at (4,0) {};
        \node[set] (Z) at (8,0) {};
        \node at (0, 1.8) {$X$};
        \node at (4, 1.8) {$Y$};
        \node at (8, 1.8) {$Z$};
        
        % Phần tử
        \node[element, label=above:$x$] (x) at (0, 0) {};
        \node[element, label={[label distance=-1mm]above:$f(x)$}] (y) at (4, 0) {};
        \node[element, label=above:$g(f(x))$] (z) at (8, 0) {};
        
        % Mũi tên ánh xạ
        \draw[->, thick, blue!70!black] (x) to node[above] {$f$} (y);
        \draw[->, thick, blue!70!black] (y) to node[above] {$g$} (z);
        \draw[->, thick, dashed, red!80!black, bend left=-40] (x) to node[below] {$g \circ f$} (z);
        
        % Chú thích
        \node at (4, -2.4) {Ánh xạ hợp};
    \end{scope}

    \end{tikzpicture}
    \caption{Minh họa ánh xạ ngược và ánh xạ hợp.}
    \label{fig:inverse_composite}
\end{figure}
% --- Kết thúc hình minh họa ánh xạ ngược và hợp ---