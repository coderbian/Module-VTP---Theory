\begin{figure}[H]
    \centering
    \begin{tikzpicture}
        % Trục x
        \draw[->] (-3,0) -- (3,0) node[right] {};
        \node at (0,0) [below=3pt] {$a$};
        \draw (0, -0.1) -- (0, 0.1);
        \draw[red] (-1.5, 0.1) -- (-1.5, -0.1) node[below] {$a-\delta$};
        \draw[red] (1.5, 0.1) -- (1.5, -0.1) node[below] {$a+\delta$};
        \draw[red, dashed] (-1.5, 0) -- (1.5, 0);
        \node at (-0.5, 0.3) [above] {$x$};
        \filldraw[black] (-0.5,0) circle (1.5pt);

        % Trục y
        \draw[->] (5,0) -- (11,0) node[right] {};
        \node at (8,0) [below=3pt] {$L$};
        \draw (8, -0.1) -- (8, 0.1);
        \draw[red] (6, 0.1) -- (6, -0.1) node[below] {$L-\epsilon$};
        \draw[red] (10, 0.1) -- (10, -0.1) node[below] {$L+\epsilon$};
        \draw[red, dashed] (6, 0) -- (10, 0);
        \node at (8.8, 0.3) [above] {$f(x)$};
        \filldraw[black] (8.8,0) circle (1.5pt);
        
        % Ánh xạ f
        \draw[->, thick, blue!60] (-0.5, 0) .. controls (4, 2.5) and (7, 2.5) .. (8.8, 0) node[midway, above] {$f$};
    \end{tikzpicture}
    \caption{Minh họa định nghĩa $\epsilon-\delta$ của giới hạn.}
    \label{fig:epsilon-delta}
\end{figure}