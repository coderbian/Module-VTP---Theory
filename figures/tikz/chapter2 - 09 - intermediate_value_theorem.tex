\begin{figure}[H]
	\centering
	\begin{tikzpicture}
		\begin{axis}[
			axis lines=middle,
			xtick=\empty,
			ytick=\empty,
			xlabel={},
			ylabel={},
			xmin=-0.5, xmax=5.5,
			ymin=-0.5, ymax=5.5,
			clip=false, % Allow drawing outside the axis box
			]
			% Define coordinates for the labels
			\coordinate (fa_label) at (axis cs:0, 4.6);
			\coordinate (fb_label) at (axis cs:0, 1);
			\coordinate (N_label) at (axis cs:0, 2.5);
			
			% Draw the function y=f(x)
			\addplot[
			domain=1:5,
			samples=100,
			smooth,
			color=teal, % A greenish-blue color
			thick,
			] {3 - 0.5*(x-1) - 0.5*sin(180*(x-1)) + 0.2*(x-3)^2 - 0.1*(x-3)^3};
			\node[above right, color=teal] at (axis cs:1.8, 3.5) {\( y = f(x) \)};
			
			% Draw the horizontal line y=N
			\addplot[
			domain=-0.5:5.5,
			color=blue,
			] {2.5};
			\node[above right, color=blue] at (axis cs:3.5, 2.5) {\( y = N \)};
			
			% Draw dashed lines from x-axis and y-axis to the curve points
			\draw[dashed, red] (fa_label) -- (axis cs:1, 4.6);
                \draw[dashed, blue] (axis cs: 1, 0) -- (axis cs: 1, 4.6)
			\draw[dashed, red] (fb_label) -- (axis cs:5, 1);
                \draw[dashed, blue] (axis cs: 5, 0) -- (axis cs: 5, 1)
			
			% Add labels on the y-axis
			\node[left] at (fa_label) {\( f(a) \)};
			\node[left] at (fb_label) {\( f(b) \)};
			\node[above left] at (N_label) {\( N \)};
			
			% Add points and labels on the x-axis
			\fill[blue] (axis cs:1, 0) circle (2pt) node[below=2pt] {\( a \)};
			\fill[blue] (axis cs:5, 0) circle (2pt) node[below=2pt] {\( b \)};
		\end{axis}
	\end{tikzpicture}
	\caption{Minh họa hình học định lý giá trị trung gian \footnotemark}
	\label{fig:intermediate_value_theorem}
\end{figure}
\footnotetext{dường nằm ngang \( y = N \) nằm giữa \( y = f(a) \) và \( y = f(b) \) luôn cắt đồ thị hàm số \( f \) ở ít nhất một điểm.}