
% --- Bắt đầu môi trường hình ảnh ---
\begin{figure}[!h]
    \centering % Căn giữa hình ảnh
    \begin{tikzpicture}[
        % --- CÁC THIẾT LẬP CHUNG (BẠN CÓ THỂ CHỈNH SỬA Ở ĐÂY) ---
        scale=0.85, % Tỉ lệ tổng thể của hình vẽ
        transform shape,
        % Định nghĩa style cho các tập hợp (hình ellipse)
        set/.style={
            rectangle, 
            draw, % Vẽ đường viền
            minimum width=2.5cm, 
            minimum height=4cm,
            line width=0.5pt % Độ dày đường viền
        },
        % Định nghĩa style cho các phần tử (chấm tròn)
        element/.style={
            circle, 
            fill, % Tô màu brown
            brown,
            inner sep=1.5pt % Kích thước của chấm
        },
        % Định nghĩa style cho các mũi tên
        arrow/.style={
            ->, % Kiểu mũi tên
            thick, % Độ dày mũi tên
            >=stealth % Kiểu đầu mũi tên
        }
    ]

    % --- HÀNG 1, CỘT 1: ÁNH XẠ THƯỜNG ---
    \begin{scope}[xshift=0cm, yshift=0cm]
        % Vẽ tập nguồn X và tập đích Y
        \node[set] (X1) at (0,0) {};
        \node[set] (Y1) at (4,0) {};
        \node at (0, 2.5) {$X$}; % Nhãn cho tập X
        \node at (4, 2.5) {$Y$}; % Nhãn cho tập Y
        
        % Các phần tử trong tập X
        \node[element] (x1a) at (0, 1) {};
        \node[element] (x1b) at (0, 0) {};
        \node[element] (x1c) at (0, -1) {};

        % Các phần tử trong tập Y
        \node[element] (y1a) at (4, 1.5) {};
        \node[element] (y1b) at (4, 0) {};
        \node[element] (y1c) at (4, -1.5) {};
        
        % Vẽ các mũi tên ánh xạ
        \draw[arrow] (x1a) to (y1a);
        \draw[arrow] (x1b) to (y1b);
        \draw[arrow] (x1c) to (y1b); % Hai phần tử cùng trỏ đến y1b
        
        % Chú thích cho hình
        \node at (2, -2.8) {\textbf{Ánh xạ}};
    \end{scope}

    % --- HÀNG 1, CỘT 2: ĐƠN ÁNH ---
    \begin{scope}[xshift=8cm, yshift=0cm]
        % Vẽ tập nguồn X và tập đích Y
        \node[set] (X2) at (0,0) {};
        \node[set] (Y2) at (4,0) {};
        \node at (0, 2.5) {$X$};
        \node at (4, 2.5) {$Y$};
        
        % Các phần tử trong tập X
        \node[element] (x2a) at (0, 1) {};
        \node[element] (x2b) at (0, -1) {};

        % Các phần tử trong tập Y
        \node[element] (y2a) at (4, 1.5) {};
        \node[element] (y2b) at (4, 0) {};
        \node[element] (y2c) at (4, -1.5) {};
        
        % Vẽ các mũi tên ánh xạ
        \draw[arrow] (x2a) to (y2a);
        \draw[arrow] (x2b) to (y2c); % Mỗi phần tử X trỏ đến 1 phần tử Y riêng biệt
        
        % Chú thích cho hình
        \node at (2, -2.8) {\textbf{Đơn ánh}};
    \end{scope}

    % --- HÀNG 2, CỘT 1: TOÀN ÁNH ---
    \begin{scope}[xshift=0cm, yshift=-6cm]
        % Vẽ tập nguồn X và tập đích Y
        \node[set] (X3) at (0,0) {};
        \node[set] (Y3) at (4,0) {};
        \node at (0, 2.5) {$X$};
        \node at (4, 2.5) {$Y$};
        
        % Các phần tử trong tập X
        \node[element] (x3a) at (0, 1.5) {};
        \node[element] (x3b) at (0, 0) {};
        \node[element] (x3c) at (0, -1.5) {};

        % Các phần tử trong tập Y
        \node[element] (y3a) at (4, 1) {};
        \node[element] (y3b) at (4, -1) {};
        
        % Vẽ các mũi tên ánh xạ
        \draw[arrow] (x3a) to (y3a);
        \draw[arrow] (x3b) to (y3b);
        \draw[arrow] (x3c) to (y3a); % Mọi phần tử Y đều có tiền ảnh
        
        % Chú thích cho hình
        \node at (2, -2.8) {\textbf{Toàn ánh}};
    \end{scope}

    % --- HÀNG 2, CỘT 2: SONG ÁNH ---
    \begin{scope}[xshift=8cm, yshift=-6cm]
        % Vẽ tập nguồn X và tập đích Y
        \node[set] (X4) at (0,0) {};
        \node[set] (Y4) at (4,0) {};
        \node at (0, 2.5) {$X$};
        \node at (4, 2.5) {$Y$};
        
        % Các phần tử trong tập X
        \node[element] (x4a) at (0, 1.5) {};
        \node[element] (x4b) at (0, 0) {};
        \node[element] (x4c) at (0, -1.5) {};

        % Các phần tử trong tập Y
        \node[element] (y4a) at (4, 1.5) {};
        \node[element] (y4b) at (4, 0) {};
        \node[element] (y4c) at (4, -1.5) {};
        
        % Vẽ các mũi tên ánh xạ
        \draw[arrow] (x4a) to (y4a);
        \draw[arrow] (x4b) to (y4b);
        \draw[arrow] (x4c) to (y4c); % Tương ứng 1-1
        
        % Chú thích cho hình
        \node at (2, -2.8) {\textbf{Song ánh}};
    \end{scope}

    \end{tikzpicture}
    \caption{Minh họa các loại ánh xạ khác nhau.}
    \label{fig:mapping_types_grid}
\end{figure}
% --- Kết thúc môi trường hình ảnh ---