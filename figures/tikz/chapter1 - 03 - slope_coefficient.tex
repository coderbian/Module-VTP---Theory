\begin{figure}[htbp]
    \centering
    \begin{tikzpicture}[
        scale=0.8, 
        font=\small,
        dot/.style={circle, fill, inner sep=1.2pt}
    ]
        % Load necessary libraries
        \usetikzlibrary{decorations.pathreplacing, calligraphy}

        % Vẽ hệ trục tọa độ
        \draw[->, thick] (-4.5,0) -- (4.5,0) node[below left] {$x$};
        \draw[->, thick] (0,-4) -- (0,4.5) node[below left] {$y$};

        % Vẽ đường thẳng y = 1.2x + 0.5
        \draw[thick, blue] (-3.5, -3.7) -- (3, 4.1);

        % --- Tam giác #1 (dưới cùng, bên trái) ---
        \coordinate (x1) at (-3, -3.1);
        \coordinate (x1_plus_1) at (-2, -1.9);
        \coordinate (c1) at (-2, -3.1);
        
        \draw[fill=brown!20, pattern=north east lines, pattern color=brown!60] 
            (x1) -- (x1_plus_1) -- (c1) -- cycle;
        \draw[thick] (x1) -- (x1_plus_1) -- (c1) -- cycle;
        
        \node[dot, label=left:{$(x_1, y_1)$}] at (x1) {};
        \node[dot, label=above left:{$(x_1+1, y'_1)$}] at (x1_plus_1) {};
        
        \draw[decorate, decoration={calligraphic brace, amplitude=4pt, mirror}] 
            (c1) -- (x1_plus_1) node[midway, xshift=24pt] {$y'_1 - y_1$};
        \draw[decorate, decoration={calligraphic brace, amplitude=4pt, mirror}] 
            (x1) -- (c1) node[midway, yshift=-12pt] {$1$};
        \node at ($(c1)+(-0.3,0.3)$) {};

        % --- Tam giác #3 (trên cùng, bên phải) ---
        \coordinate (x2) at (0.5, 1.1);
        \coordinate (x3) at (2.5, 3.5);
        \coordinate (c2) at (2.5, 1.1);

        \draw[fill=purple!20, pattern=north east lines, pattern color=purple!60] 
            (x2) -- (x3) -- (c2) -- cycle;
        \draw[thick] (x2) -- (x3) -- (c2) -- cycle;

        \node[dot, label=left:{$(x_2, y_2)$}] at (x2) {};
        \node[dot, label=above left:{$(x_3, y_3)$}] at (x3) {};

        \draw[decorate, decoration={calligraphic brace, amplitude=4pt, mirror}] 
            (c2) -- (x3) node[midway, xshift=24pt] {$y_3 - y_2$};
        \draw[decorate, decoration={calligraphic brace, amplitude=4pt}] 
            (c2) -- (x2) node[midway, yshift=-12pt] {$x_3 - x_2$};
        \node at ($(c2)+(-0.3,0.3)$) {};

    \end{tikzpicture}
    \caption{\centering Hệ số góc của đường thẳng không phụ thuộc vào cách chọn hai điểm để tính, tương thích với tính chất tam giác đồng dạng của hình học Euclid.}
    \label{fig:he_so_goc_v2}
\end{figure}