% preamble.tex
\documentclass[12pt, a4paper, twoside, openany]{book}
% -----------------------------
% Hỗ trợ tiếng Việt cho pdfLaTeX
% -----------------------------
\usepackage{fontspec} 
\setmainfont{Times New Roman}
\usepackage{xcolor}
\usepackage[most]{tcolorbox}
\usepackage[vietnam]{babel}
\usepackage[explicit]{titlesec}
\usepackage{tikz}
\usepackage{tkz-tab}
\usetikzlibrary{shapes.geometric}
\usepackage{pgfplots}
\pgfplotsset{compat=1.17} % Cài đặt phiên bản tương thích cho pgfplots
% \usepackage[svgnames]{xcolor} % Thêm nhiều màu sắc hơn
\usepackage{lmodern}% to have a large font size for the chapter numbers
\usepackage{lipsum}% just to generate text for the example
\usepackage{enumitem}
\usepackage{etoolbox} % Gói này cung cấp công cụ để tùy chỉnh các môi trường có sẵn
\definecolor{myblue}{RGB}{0,0,255}
\definecolor{chapnumberbg}{RGB}{26,40,105}
\definecolor{chapname}{RGB}{100,117,158}

\usepackage{background}

\backgroundsetup{
  scale=1,
  angle=54.7,
  opacity=0.1,                    % độ mờ watermark
  position={current page.center},  % đặt gần giữa trang (thay đổi bằng vshift)
  vshift=0pt,
  contents={%
    \makebox[\textwidth]{%
      \hfill
      % % logo (center vertically)
      % \raisebox{-0.5\height}{\includegraphics[width=5cm]{figures/FirstByteLogoV2.png}}%
      % \hspace{2.5em}%
      % text (center vertically)
      \raisebox{-0.5\height}{\sffamily\bfseries\fontsize{48}{30}\selectfont\color{black!40} FIRSTBYTE - Where Knowledge Begin}%
      \hfill
    }%
  }
}

% -----------------------------
% Gói toán học
% -----------------------------
\usepackage{amsmath,amssymb,amsthm,mathtools}

% -----------------------------
% Gói trình bày
% -----------------------------
\usepackage{geometry}
\geometry{
  top=2.5cm,
  bottom=3cm,
  left=2.5cm,
  right=2.5cm
}

\usepackage{graphicx}
\usepackage{hyperref}
\usepackage{bookmark}
\usepackage{fancyhdr}
\usepackage{float}

% Header / Footer
\pagestyle{fancy}
\fancyhf{}
\fancyhead[LE,RO]{\thepage}
\fancyhead[RE]{\nouppercase{\leftmark}}
\fancyhead[LO]{\nouppercase{\rightmark}}
\setlength{\headheight}{14.49998pt}
\fancyfoot[LE, RO]{\text{FIRSTBYTE - Where Knowledge Begins}} % Footer bên trái
\fancyfoot[RE, LO]{\text{Trần Minh Hiếu Học}}
\renewcommand{\footrulewidth}{0.4pt} % THÊM ĐƯỜNG KẺ FOOTER

% --------- Đánh số section ---------
\numberwithin{section}{chapter} % section = 1,2,... theo chương
\numberwithin{subsection}{section}
\numberwithin{subsubsection}{subsection}

% -----------------------------
% --------- Format tiêu đề section/subsection ---------
\titleformat{\section}[hang]
  {\normalfont\LARGE\bfseries\color{myblue}}
  {\thesection}{1em}{#1}

\titleformat{\subsection}[hang]
  {\normalfont\Large\bfseries\color{myblue}}
  {\thesubsection}{1em}{#1}

\titleformat{\subsubsection}[hang]
  {\normalfont\large\bfseries\color{myblue}}
  {\thesubsubsection}{1em}{#1}



\titleformat{\chapter}[display]
  {\normalfont}
  {}
  {0pt}
  {%
    \begin{tikzpicture}
    \node[
      draw=chapname,
      rounded corners,
      outer sep=0pt,
      inner sep=6pt,
      rotate=90,
      line width=1pt,
      font=\Large\color{chapnumberbg}\bfseries
      ]
      (chapname) 
      {\chaptertitlename};
    \node[
      fill=chapnumberbg,
      minimum width=2cm,
      minimum height=2.3cm,
      rounded corners,
      anchor=west,
      font=\color{white}\fontsize{40}{48}\selectfont\bfseries
      ]
      at ([xshift=6pt]chapname.south)
      (chapnumber)
      {\thechapter};
    \node[
      anchor=west,
      text width=\textwidth-4cm,
      font=\bfseries\LARGE
      ] 
      at ([xshift=10pt]chapnumber.east)
      {#1};
    \fill[
      overlay,
      draw=none,
      line width=0pt,
      rounded corners=1pt,
      left color=chapnumberbg,
      right color=chapnumberbg!10
      ]
      ([yshift=-3pt]chapname.north west) rectangle ++(\textwidth,-3pt);  
    \end{tikzpicture}%
  }
\titleformat{name=\chapter,numberless}[display]
  {\normalfont}
  {}
  {0pt}
  {%
    \begin{tikzpicture}
    \node[
      anchor=west,
      inner sep=0pt,
      outer sep=0pt,
      text width=\textwidth,
      font=\bfseries\LARGE
      ]
      (chaptitle) 
      {#1};
    \fill[
      overlay,
      draw=none,
      line width=0pt,
      rounded corners=1pt,
      left color=chapnumberbg,
      right color=chapnumberbg!10
      ]
      ([yshift=-3pt]chaptitle.south west) rectangle ++(\textwidth,-3pt);  
    \end{tikzpicture}%
  }


\renewcommand{\contentsname}{Mục lục}
\renewcommand{\listfigurename}{Danh sách hình vẽ}
\renewcommand{\listtablename}{Danh sách bảng biểu}
\renewcommand{\chaptername}{Chương}
\renewcommand{\bibname}{Tài liệu tham khảo}
\renewcommand{\indexname}{Chỉ mục}
\renewcommand{\figurename}{Hình}
\newcommand{\version}{v1.0 - \today}

% -----------------------------
% Định nghĩa môi trường
% -----------------------------
\newtheorem{theorem}{Định lý}[chapter]
\newtheorem{lemma}{Bổ đề}[chapter]
\newtheorem{proposition}{Mệnh đề}[chapter]
\newtheorem{corollary}{Hệ quả}[chapter]


\theoremstyle{definition}
\newtheorem{definition}{Định nghĩa}[chapter]
\newtheorem{example}{Ví dụ}[chapter]
\newtheorem{exercise}{Bài tập}[chapter]

\renewcommand{\proofname}{Chứng minh}
\newenvironment{solution}{\par\noindent\textit{Lời giải.} }{\qed}

\newtcolorbox{importantbox}{
  colback=yellow!10!white,
  colframe=blue!75!black,
  boxrule=0.5pt,
  arc=2mm,
  center
}

\newtcolorbox[auto counter, number within=chapter]{mynote}[1][]{
  colback=blue!10!white,
  colframe=black!75!blue,
  boxrule=0.5pt,
  arc=2mm,
  fonttitle=\bfseries,
  title={Ghi chú~\thetcbcounter}, % Tự động thêm tiêu đề "Ghi chú" và đánh số
  #1
}

% Tự động bỏ thụt đầu dòng cho đoạn văn bản ngay sau các môi trường
\makeatletter
\AfterEndEnvironment{theorem}{\@afterindentfalse}
\AfterEndEnvironment{lemma}{\@afterindentfalse}
\AfterEndEnvironment{proposition}{\@afterindentfalse}
\AfterEndEnvironment{corollary}{\@afterindentfalse}
\AfterEndEnvironment{definition}{\@afterindentfalse}
\AfterEndEnvironment{example}{\@afterindentfalse}
\AfterEndEnvironment{exercise}{\@afterindentfalse}
\AfterEndEnvironment{proof}{\@afterindentfalse}
\AfterEndEnvironment{solution}{\@afterindentfalse}
\makeatother

% -----------------------------
% Gọi file macros
% -----------------------------
% macros.tex

% Số học cơ bản
\newcommand{\N}{\mathbb{N}}   % Tập số tự nhiên
\newcommand{\Z}{\mathbb{Z}}   % Tập số nguyên
\newcommand{\Q}{\mathbb{Q}}   % Tập số hữu tỉ
\newcommand{\R}{\mathbb{R}}   % Tập số thực
\newcommand{\C}{\mathbb{C}}   % Tập số phức

% Đại số tuyến tính
\newcommand{\vect}[1]{\mathbf{#1}} % vector in bold
\newcommand{\mat}[1]{\mathbf{#1}}  % matrix in bold
\newcommand{\T}{^{\mathsf{T}}}     % chuyển vị
\DeclareMathOperator{\rank}{rank}
\DeclareMathOperator{\diag}{diag}
\DeclareMathOperator{\Ker}{Ker}
% \DeclareMathOperator{\Im}{Im}

% Phân tích / Giải tích
\newcommand{\dd}{\,\mathrm{d}}           % vi phân
\newcommand{\deriv}[2]{\frac{\mathrm{d}#1}{\mathrm{d}#2}} % đạo hàm
\newcommand{\pderiv}[2]{\frac{\partial #1}{\partial #2}}  % đạo hàm riêng
\newcommand{\limit}[2]{\lim\limits_{#1 \to #2}}           % giới hạn
\newcommand{\intt}[4]{\int_{#1}^{#2} #3 \,\mathrm{d}#4}   % tích phân có cận

% Ký hiệu khác
\newcommand{\norm}[1]{\left\lVert #1 \right\rVert}   % chuẩn
\newcommand{\abs}[1]{\left\lvert #1 \right\rvert}    % giá trị tuyệt đối
\newcommand{\inner}[2]{\left\langle #1, #2 \right\rangle} % tích vô hướng

