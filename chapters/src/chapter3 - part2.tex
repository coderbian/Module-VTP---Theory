\section{Các công thức cho đạo hàm}
\subsection{Đạo hàm của hàm hợp}

Giả sử ta muốn tính đạo hàm của hàm số $y = \sin(x^3 + 2x)$. Ta có thể nhận thấy đây là hàm hợp của hai hàm: $y = \sin(u)$ và $u = x^3 + 2x$. Ta đã biết cách tính $\deriv{y}{u} = \cos(u)$ và $\deriv{u}{x} = 3x^2 + 2$. Câu hỏi đặt ra là, làm thế nào để kết hợp hai kết quả này để tìm ra đạo hàm của $y$ theo $x$?

Một cách trực quan, ta có thể xem xét tỉ lệ thay đổi. Nếu $y$ thay đổi theo $u$ và $u$ thay đổi theo $x$, thì tỉ lệ thay đổi của $y$ so với $x$ có thể được hình dung như là tích của các tỉ lệ thay đổi thành phần. Sử dụng kí hiệu vi phân của Leibniz, ta có thể viết một cách hình thức:
$$
\dfrac{\Delta y}{\Delta x} = \dfrac{\Delta y}{\Delta u} \cdot \dfrac{\Delta u}{\Delta x}
$$
Khi cho $\Delta x \to 0$, vì $u$ là hàm khả vi theo $x$ nên nó cũng liên tục, do đó $\Delta u \to 0$. Lấy giới hạn hai vế, ta có thể dự đoán công thức:
$$
\deriv{y}{x} = \deriv{y}{u} \cdot \deriv{u}{x}
$$
Tuy nhiên, lý luận này có một lỗ hổng: nó không hoạt động nếu $\Delta u = 0$ với các giá trị $\Delta x$ khác không tùy ý gần 0. Dưới đây là một cách chứng minh chặt chẽ hơn.

\begin{theorem}[Quy tắc đạo hàm hàm hợp\footnote{Còn được gọi là \textbf{quy tắc xích}, trong tiếng Anh là \textit{chain rule}.}]
Nếu hàm số $g$ khả vi tại $x$ và hàm số $f$ khả vi tại $g(x)$ thì hàm hợp $f \circ g$ khả vi tại $x$ và
\begin{tcolorbox}[colback=yellow!10!white, colframe=blue!75!black, boxrule=0.5pt, arc=2mm]
$$
(f \circ g)'(x) = f'(g(x)) \cdot g'(x)
$$
\end{tcolorbox}
\end{theorem}
\begin{proof}
Trước hết, ta thiết lập một biểu diễn cho số gia của hàm số. Với một hàm $f$ bất kỳ khả vi tại $a$, ta có
$$
\limit{\Delta x}{0} \dfrac{f(a + \Delta x) - f(a)}{\Delta x} = f'(a)
$$
Đặt $\Delta y = f(a + \Delta x) - f(a)$ và định nghĩa một hàm sai số $\varepsilon(\Delta x) = \dfrac{\Delta y}{\Delta x} - f'(a)$. Rõ ràng $\limit{\Delta x}{0} \varepsilon(\Delta x) = 0$. Từ đó, ta có thể viết:
$$
\Delta y = [f'(a) + \varepsilon(\Delta x)] \Delta x
$$
Bây giờ, ta áp dụng nhận xét này cho bài toán của chúng ta.
Đặt $u = g(x)$ và $y = f(u)$.
Số gia của $u$ khi $x$ thay đổi một lượng $\Delta x$ là $\Delta u = g(x + \Delta x) - g(x)$. Vì $g$ khả vi tại $x$, ta có:
$$
\Delta u = [g'(x) + \varepsilon_1(\Delta x)] \Delta x, \quad \text{với } \limit{\Delta x}{0} \varepsilon_1(\Delta x) = 0.
$$
Tương tự, số gia của $y$ khi $u$ thay đổi một lượng $\Delta u$ là $\Delta y = f(g(x) + \Delta u) - f(g(x))$. Vì $f$ khả vi tại $g(x)$, ta có:
$$
\Delta y = [f'(g(x)) + \varepsilon_2(\Delta u)] \Delta u, \quad \text{với } \limit{\Delta u}{0} \varepsilon_2(\Delta u) = 0.
$$
Kết hợp hai đẳng thức trên, ta được:
\begin{align*}
\Delta y &= [f'(g(x)) + \varepsilon_2(\Delta u)] \cdot [g'(x) + \varepsilon_1(\Delta x)] \Delta x \\
\implies \dfrac{\Delta y}{\Delta x} &= [f'(g(x)) + \varepsilon_2(\Delta u)] \cdot [g'(x) + \varepsilon_1(\Delta x)]
\end{align*}
Khi cho $\Delta x \to 0$, ta có $\Delta u \to 0$ (vì $g$ liên tục tại $x$). Do đó, $\varepsilon_1(\Delta x) \to 0$ và $\varepsilon_2(\Delta u) \to 0$. Lấy giới hạn hai vế, ta được:
$$
\limit{\Delta x}{0} \dfrac{\Delta y}{\Delta x} = [f'(g(x)) + 0] \cdot [g'(x) + 0] = f'(g(x)) \cdot g'(x)
$$
Đây chính là công thức đạo hàm của hàm hợp.
\end{proof}

\begin{example}
Tính đạo hàm của hàm số $y = (x^4 + 3x)^ {50}$.
\end{example}
\begin{solution}
Ta xem đây là hàm hợp với $y = u^{50}$ và $u = x^4 + 3x$. Áp dụng quy tắc xích:
$$
\deriv{y}{x} = \deriv{y}{u} \cdot \deriv{u}{x} = (50u^{49}) \cdot (4x^3 + 3)
$$
Thay $u = x^4 + 3x$ trở lại, ta được:
$$
\deriv{y}{x} = 50(x^4 + 3x)^{49} (4x^3 + 3)
$$
\end{solution}

\begin{example}
Tính đạo hàm của $h(x) = \sin(x^3 + 2x)$.
\end{example}
\begin{solution}
Đặt $f(x) = \sin(x)$ và $g(x) = x^3 + 2x$. Ta có $f'(x) = \cos(x)$ và $g'(x) = 3x^2+2$.
Áp dụng công thức đạo hàm hàm hợp $(f \circ g)'(x) = f'(g(x))g'(x)$, ta được:
$$
h'(x) = \cos(x^3+2x) \cdot (3x^2+2)
$$
\end{solution}

\begin{example}
    Tính đạo hàm của hàm số $f(x) = \cos^5(x^3 - 4x^2 + 1)$.
\end{example}
\begin{solution}
    Đây là một ví dụ phức tạp hơn, đòi hỏi áp dụng quy tắc xích nhiều lần. Ta có thể phân tích hàm số này thành các lớp:
    \begin{itemize}
        \item Lớp ngoài cùng: $y = u^5$ với $u = \cos(v)$
        \item Lớp giữa: $u = \cos(v)$ với $v = x^3 - 4x^2 + 1$
        \item Lớp trong cùng: $v = x^3 - 4x^2 + 1$
    \end{itemize}
    Áp dụng quy tắc xích từ ngoài vào trong:
    \begin{align*}
        f'(x) &= 5\cos^4(x^3 - 4x^2 + 1) \cdot \deriv{}{x}\left( \cos(x^3 - 4x^2 + 1) \right) \\
        &= 5\cos^4(x^3 - 4x^2 + 1) \cdot \left( -\sin(x^3 - 4x^2 + 1) \right) \cdot \deriv{}{x}(x^3 - 4x^2 + 1) \\
        &= -5\cos^4(x^3 - 4x^2 + 1) \sin(x^3 - 4x^2 + 1) (3x^2 - 8x).
    \end{align*}
\end{solution}

Từ các ví dụ trên, ta có thể tổng quát hóa một công thức rất hữu ích, thường được gọi là \textbf{Quy tắc lũy thừa tổng quát}.

\begin{proposition}[Quy tắc lũy thừa tổng quát]
Nếu $k$ là một số nguyên dương và $g(x)$ là một hàm khả vi, thì
\begin{tcolorbox}[colback=yellow!10!white, colframe=blue!75!black, boxrule=0.5pt, arc=2mm]
$$
\deriv{}{x} [g(x)]^k = k \cdot [g(x)]^{k-1} \cdot g'(x)
$$
\end{tcolorbox}
\end{proposition}
\begin{proof}
Đây là một trường hợp đặc biệt của quy tắc xích. Đặt $f(u) = u^k$, khi đó $f'(u) = k u^{k-1}$. Áp dụng quy tắc xích cho hàm hợp $f(g(x))$, ta có:
$$
\deriv{}{x} f(g(x)) = f'(g(x)) \cdot g'(x) = k [g(x)]^{k-1} \cdot g'(x).
$$
\end{proof}

\subsection{Đạo hàm của hàm ngược}

Nếu hàm số $y=f(x)$ có hàm ngược là $x = g(y)$, thì theo định nghĩa của hàm ngược, ta có mối liên hệ:
$$
g(f(x)) = x
$$
Giả sử cả hai hàm $f$ và $g$ đều khả vi. Lấy đạo hàm hai vế của đẳng thức trên theo biến $x$ và áp dụng quy tắc xích, ta được:
$$
g'(f(x)) \cdot f'(x) = 1
$$
Vì $y = f(x)$, ta có thể viết lại $g'(f(x))$ thành $g'(y)$. Do đó:
$$
g'(y) \cdot f'(x) = 1
$$
Từ đây, ta có thể suy ra một công thức đơn giản và đẹp mắt cho đạo hàm của hàm ngược:
$$
g'(y) = \dfrac{1}{f'(x)}
$$
Một cách tiếp cận khác mang tính trực quan hơn là sử dụng kí hiệu vi phân. Ta có thể viết một cách hình thức:
$$
\dfrac{\Delta x}{\Delta y} = \dfrac{1}{\frac{\Delta y}{\Delta x}}
$$
Giả sử rằng hàm ngược $g$ là liên tục, khi đó $\Delta y \to 0$ sẽ kéo theo $\Delta x \to 0$. Lấy giới hạn hai vế, ta thu được:
$$
\deriv{x}{y} = \dfrac{1}{\deriv{y}{x}}
$$
Cả hai lý luận trên đều giả định trước sự khả vi hoặc liên tục của hàm ngược. Dưới đây là một phát biểu tổng quát và chứng minh đầy đủ hơn.

\begin{theorem}[Đạo hàm của hàm ngược]
Giả sử $f: (a, b) \to (c, d)$ là một song ánh liên tục và $f^{-1}$ là hàm ngược của $f$. Nếu $f$ có đạo hàm tại $x \in (a, b)$ và $f'(x) \ne 0$, thì $f^{-1}$ có đạo hàm tại $y = f(x)$, và
\begin{tcolorbox}[colback=yellow!10!white, colframe=blue!75!black, boxrule=0.5pt, arc=2mm]
$$
(f^{-1})'(y) = \dfrac{1}{f'(x)}
$$
\end{tcolorbox}
\end{theorem}
\begin{proof}
Cho $y \in (c, d)$ và xét một số gia $\Delta y$ đủ nhỏ sao cho $y + \Delta y \in (c, d)$. Đặt $x = f^{-1}(y)$ và $\Delta x = f^{-1}(y + \Delta y) - f^{-1}(y)$.
Từ đó suy ra $f^{-1}(y + \Delta y) = x + \Delta x$, và do đó $\Delta y = f(x + \Delta x) - f(x)$.

Theo định nghĩa đạo hàm, ta có:
\begin{align*}
(f^{-1})'(y) &= \limit{\Delta y}{0} \dfrac{f^{-1}(y + \Delta y) - f^{-1}(y)}{\Delta y} \\
&= \limit{\Delta y}{0} \dfrac{\Delta x}{f(x + \Delta x) - f(x)} \\
&= \limit{\Delta y}{0} \dfrac{1}{\frac{f(x + \Delta x) - f(x)}{\Delta x}}
\end{align*}
Vì $f$ là một song ánh liên tục trên một khoảng, ta có thể chứng minh được rằng hàm ngược $f^{-1}$ cũng liên tục (chứng minh chi tiết có thể tham khảo trong [TPTT02, tr. 60] hoặc [Spi94, tr. 232]). Do tính liên tục của $f^{-1}$, khi $\Delta y \to 0$ thì $\Delta x = f^{-1}(y + \Delta y) - f^{-1}(y) \to 0$.

Vì vậy, ta có thể đổi biến trong giới hạn:
$$
(f^{-1})'(y) = \limit{\Delta x}{0} \dfrac{1}{\frac{f(x + \Delta x) - f(x)}{\Delta x}} = \dfrac{1}{\limit{\Delta x}{0} \frac{f(x + \Delta x) - f(x)}{\Delta x}} = \dfrac{1}{f'(x)}.
$$
\end{proof}

Công thức đạo hàm của hàm ngược là một công cụ mạnh mẽ, đặc biệt hữu ích khi tính đạo hàm của các hàm sơ cấp như hàm logarit và hàm lượng giác ngược, như chúng ta sẽ thấy ở mục tiếp theo.

\subsection{Đạo hàm của các hàm số sơ cấp}

Trong phần này, chúng ta sẽ xây dựng một bộ công thức đạo hàm cho các hàm số quen thuộc, bắt đầu với các hàm lượng giác, hàm mũ, logarit và hàm lũy thừa.

\subsubsection{Đạo hàm của các hàm lượng giác}

\begin{example}[Đạo hàm của hàm sin]
    Ta sẽ tính đạo hàm của hàm số $y = \sin x$ bằng định nghĩa.
    \begin{align*}
        (\sin x)' &= \limit{h}{0} \dfrac{\sin(x+h) - \sin x}{h} \\
        &= \limit{h}{0} \dfrac{\sin x \cos h + \cos x \sin h - \sin x}{h} \\
        &= \limit{h}{0} \left( \sin x \dfrac{\cos h - 1}{h} + \cos x \dfrac{\sin h}{h} \right) \\
        &= \sin x \left( \limit{h}{0} \dfrac{\cos h - 1}{h} \right) + \cos x \left( \limit{h}{0} \dfrac{\sin h}{h} \right).
    \end{align*}
    Chúng ta đã biết hai giới hạn cơ bản: $\limit{h}{0} \dfrac{\sin h}{h} = 1$ và $\limit{h}{0} \dfrac{\cos h - 1}{h} = 0$. Thay các giá trị này vào biểu thức trên, ta được:
    \[ (\sin x)' = \sin x \cdot 0 + \cos x \cdot 1 = \cos x. \]
    Vậy, ta có công thức đạo hàm đầu tiên cho hàm lượng giác:
    \begin{importantbox}
        \[ (\sin x)' = \cos x. \]
    \end{importantbox}
\end{example}

\begin{example}[Đạo hàm của hàm cos]
    Để tính đạo hàm của hàm $\cos x$, ta có thể sử dụng mối liên hệ $\cos x = \sin\left(\dfrac{\pi}{2} - x\right)$ và áp dụng quy tắc đạo hàm của hàm hợp:
    \[ (\cos x)' = \left( \sin\left(\dfrac{\pi}{2} - x\right) \right)' = \cos\left(\dfrac{\pi}{2} - x\right) \cdot \left(\dfrac{\pi}{2} - x\right)' = \sin x \cdot (-1) = -\sin x. \]
    Do đó, ta có công thức:
    \begin{importantbox}
        \[ (\cos x)' = -\sin x. \]
    \end{importantbox}
\end{example}

\begin{example}[Đạo hàm của hàm tan]
    Sử dụng quy tắc đạo hàm của một thương cho hàm $\tan x = \dfrac{\sin x}{\cos x}$:
    \begin{align*}
        (\tan x)' &= \dfrac{(\sin x)' \cos x - \sin x (\cos x)'}{\cos^2 x} \\ 
                  &= \dfrac{\cos x \cdot \cos x - \sin x \cdot (-\sin x)}{\cos^2 x} \\
                  &= \dfrac{\cos^2 x + \sin^2 x}{\cos^2 x} = \dfrac{1}{\cos^2 x}
    \end{align*}
    Công thức này cũng có thể được viết dưới dạng $1 + \tan^2 x$.
    \begin{importantbox}
        \[ (\tan x)' = \dfrac{1}{\cos^2 x}. \]
    \end{importantbox}
\end{example}

\begin{example}[Đạo hàm của các hàm lượng giác ngược]
    Ta sẽ tính đạo hàm của hàm $y = \arcsin x$ và $y = \arctan x$ bằng quy tắc đạo hàm của hàm ngược.
    \begin{enumerate}[label=(\alph*)]
        \item Xét hàm $y = g(x) = \arcsin x$ với $x \in (-1, 1)$. Đây là hàm ngược của hàm $x = f(y) = \sin y$ với $y \in \left(-\dfrac{\pi}{2}, \dfrac{\pi}{2}\right)$. Áp dụng công thức, ta có:
        \[ (\arcsin x)' = g'(x) = \dfrac{1}{f'(y)} = \dfrac{1}{\cos y}. \]
        Vì $y \in \left(-\dfrac{\pi}{2}, \dfrac{\pi}{2}\right)$, $\cos y > 0$, nên $\cos y = \sqrt{1-\sin^2 y} = \sqrt{1-x^2}$.
        \begin{importantbox}
        \[ (\arcsin x)' = \dfrac{1}{\sqrt{1-x^2}},\quad x \in (-1, 1). \]
        \end{importantbox}
        \item Xét hàm $y = g(x) = \arctan x$ với $x \in \R$. Đây là hàm ngược của hàm $x = f(y) = \tan y$ với $y \in \left(-\dfrac{\pi}{2}, \dfrac{\pi}{2}\right)$.
        \[ (\arctan x)' = g'(x) = \dfrac{1}{f'(y)} = \dfrac{1}{1/\cos^2 y} = \cos^2 y. \]
        Sử dụng đồng nhất thức $1 + \tan^2 y = \dfrac{1}{\cos^2 y}$, ta có $\cos^2 y = \dfrac{1}{1+\tan^2 y} = \dfrac{1}{1+x^2}$.
        \begin{importantbox}
        \[ (\arctan x)' = \dfrac{1}{1+x^2},\quad x \in \R. \]
        \end{importantbox}
    \end{enumerate}
    Công thức đạo hàm cho hàm $\arccos x$ có thể được suy ra tương tự (Bài tập ???).
    % TODO: Sửa lại tham chiếu
    \begin{importantbox}
    \[ (\arccos x)' = -\dfrac{1}{\sqrt{1-x^2}},\quad x \in (-1, 1). \]
    \end{importantbox}
\end{example}


\subsubsection*{Đạo hàm của hàm mũ và hàm logarit}

Một trong những công thức đạo hàm thanh lịch và quan trọng nhất trong vi tích phân là đạo hàm của hàm mũ với cơ số $e$.
\begin{importantbox}
\[ (e^x)' = e^x. \]
\end{importantbox}
\textit{Giải thích.} Ta có thể phác thảo một cách chứng minh công thức này. Đặt $f(x) = e^x$. Theo định nghĩa, ta có:
\[ f'(x) = \limit{h}{0} \dfrac{e^{x+h}-e^x}{h} = e^x \cdot \limit{h}{0}\dfrac{e^h-1}{h}. \]
Giới hạn $\limit{h}{0}\dfrac{e^h-1}{h}$ chính là $f'(0)$. Bằng cách sử dụng định nghĩa của số $e$ là $\limit{u}{0} (1+u)^{1/u}$, ta có thể chứng minh được $f'(0) = 1$. Do đó, $(e^x)' = e^x$. \qed

Từ đây, ta dễ dàng suy ra đạo hàm của hàm mũ với cơ số $a$ bất kỳ ($a > 0, a \ne 1$):
\[ (a^x)' = (e^{x \ln a})' = e^{x \ln a} \cdot (x \ln a)' = a^x \ln a. \]
\begin{importantbox}
\[ (a^x)' = a^x \ln a. \]
\end{importantbox}

\begin{example}[Đạo hàm của hàm logarit]
    Xét hàm $y = g(x) = \log_a x$, là hàm ngược của $x = f(y) = a^y$. Ta có:
    \[ (\log_a x)' = g'(x) = \dfrac{1}{f'(y)} = \dfrac{1}{a^y \ln a} = \dfrac{1}{x \ln a}. \]
    \begin{importantbox}
    \[ (\log_a x)' = \dfrac{1}{x \ln a}. \]
    \end{importantbox}
    Trường hợp đặc biệt và phổ biến nhất là logarit tự nhiên:
    \begin{importantbox}
    \[ (\ln x)' = \dfrac{1}{x}. \]
    \end{importantbox}
\end{example}


\subsubsection*{Đạo hàm của hàm lũy thừa}

Đối với hàm $y=x^r$ với $r$ là một số thực bất kỳ và $x > 0$, ta có thể viết lại hàm số dưới dạng hàm mũ và logarit để tính đạo hàm:
\[ (x^r)' = (e^{r \ln x})' = e^{r \ln x} \cdot (r \ln x)' = x^r \cdot \dfrac{r}{x} = rx^{r-1}. \]
Đây là công thức tổng quát cho đạo hàm của hàm lũy thừa:
\begin{importantbox}
\[ (x^r)' = rx^{r-1}, \quad \text{với } x > 0. \]
\end{importantbox}

\begin{example}
    Với $x>0$ thì
    \[ (\sqrt{x})' = (x^{1/2})' = \dfrac{1}{2}x^{-1/2} = \dfrac{1}{2\sqrt{x}}. \]
\end{example}

\subsection{Đạo hàm của hàm ẩn}

Với một hàm số $y = f(x)$ thì giá trị của $y$ phụ thuộc theo giá trị $x$ và sự phụ thuộc này được biểu diễn một cách rõ ràng theo quy luật được cho bởi hàm $f$, ta nói $y$ là một \textbf{hàm hiện} (hay hàm tường minh) của $x$. Tuy nhiên có nhiều trường hợp $x$ và $y$ phụ thuộc nhau nhưng chúng ta không có sẵn một công thức cụ thể để biểu diễn $y$ theo $x$, khi đó người ta thường nói $y$ là \textbf{hàm ẩn} của $x$.

Việc giải hàm ẩn là giải phương trình, thường là khó. Tuy nhiên trong một số ứng dụng, mục đích của ta không phải là tìm $y$ theo $x$, mà là tìm đạo hàm của $y$ theo $x$. Để giải quyết bài toán này trong nhiều trường hợp khi hàm ẩn được cho bởi một đẳng thức giữa $x$ và $y$ ta có thể lấy đạo hàm của cả hai vế của đẳng thức theo $x$ rồi giải ra đạo hàm của $y$ theo $x$. Cụ thể hơn ta giả thiết là $y$ tồn tại trong một lân cận của mỗi giá trị của $x$ và là hàm khả vi theo $x$. Một cách sơ lược thì giả thiết này được thỏa trong trường hợp thường gặp mà đẳng thức quan hệ giữa $x$ và $y$ được cho bởi các hàm sơ cấp của $x$ và $y$. Vấn đề này được thảo luận chi tiết hơn trong môn Vi tích phân hàm nhiều biến. [Bmgt2, Chương 1]

%TODO thêm cái link cho đúng%

Đây được gọi là phương pháp \textbf{đạo hàm hàm ẩn}.

Vấn đề này rõ hơn khi ta xét ví dụ.

\begin{example}
    Cho $y$ phụ thuộc vào $x$ theo phương trình
    \[ x^3 + y^3 = 6y. \]
    Giả thiết là $y$ tồn tại trong một lân cận của mỗi giá trị của $x$ và khả vi theo $x$, hãy tính $y'(x)$.

    \begin{solution}
        Nếu muốn tính công thức hiện $y$ theo $x$ ta sẽ giải một phương trình bậc 3, một việc không dễ. Ta dùng phương pháp đạo hàm hàm ẩn. Lấy đạo hàm của cả hai vế phương trình theo $x$, nhớ rằng $y$ là một hàm của $x$ và dùng quy tắc chuỗi, ta được:
        \[ \deriv{}{x}(x^3) + \deriv{}{x}(y^3) = \deriv{}{x}(6y) \]
        \[ 3x^2 + 3y^2 \cdot y'(x) = 6y'(x). \]
        Bây giờ ta giải phương trình này để tìm $y'(x)$:
        \[ 3y^2y'(x) - 6y'(x) = -3x^2 \]
        \[ (3y^2-6)y'(x) = -3x^2, \]
        hay
        \[ y'(x) = \dfrac{-3x^2}{3y^2 - 6} = \dfrac{x^2}{2-y^2}. \]
        Ta vẫn chưa tính được đạo hàm $y'(x)$ một cách tường minh theo $x$, tuy nhiên tại mỗi điểm $(x, y)$ cụ thể cho trước trên đường cong ta có thể tìm được giá trị của $y'(x)$.

        Chẳng hạn tại điểm $(\sqrt[3]{5}, 1)$ thỏa phương trình $x^3 + y^3 = 6y$, ta có
        \[ y'(\sqrt[3]{5}) = \dfrac{(\sqrt[3]{5})^2}{2 - 1^2} = \sqrt[3]{25}. \]
    \end{solution}
\end{example}

\begin{example}
    Tìm phương trình tiếp tuyến của đường cong $x^2 + 2xy - y^2 + x = 2$ tại điểm $(1, 2)$.

    \begin{solution}
        Đầu tiên, ta kiểm tra điểm $(1,2)$ có thuộc đường cong hay không: $1^2 + 2(1)(2) - 2^2 + 1 = 1 + 4 - 4 + 1 = 2$. Vậy điểm này nằm trên đường cong.

        Tiếp theo, ta lấy đạo hàm hai vế theo $x$:
        \[ 2x + (2y + 2xy') - 2yy' + 1 = 0 \]
        Nhóm các số hạng chứa $y'$:
        \[ y'(2x - 2y) = -2x - 2y - 1 \]
        Giải tìm $y'$:
        \[ y' = \dfrac{-2x - 2y - 1}{2x - 2y} = \dfrac{2x + 2y + 1}{2y - 2x}. \]
        Tại điểm $(1, 2)$, hệ số góc của tiếp tuyến là:
        \[ m = y'(1) = \dfrac{2(1) + 2(2) + 1}{2(2) - 2(1)} = \dfrac{2+4+1}{4-2} = \dfrac{7}{2}. \]
        Phương trình tiếp tuyến với đường cong tại điểm $(1, 2)$ là:
        \[ y - 2 = \dfrac{7}{2}(x-1) \iff y = \dfrac{7}{2}x - \dfrac{3}{2}. \]
    \end{solution}
\end{example}

\begin{example}
    Cho đường cong mà các điểm $(x, y)$ trên đó thỏa phương trình
    \[ xy = \arctan\dfrac{x}{y}. \]
    Kiểm tra điểm $(\frac{\sqrt{\pi}}{2}, \frac{\sqrt{\pi}}{2})$ thuộc đường cong này. Giả sử tại gần điểm này đường cong là đồ thị của hàm số $y=y(x)$ khả vi. Viết phương trình tiếp tuyến với đường cong tại điểm này.

    \begin{solution}
        Thế tọa độ điểm $(\frac{\sqrt{\pi}}{2}, \frac{\sqrt{\pi}}{2})$ vào phương trình:
        \begin{itemize}
            \item Vế trái: $xy = \left(\dfrac{\sqrt{\pi}}{2}\right)\left(\dfrac{\sqrt{\pi}}{2}\right) = \dfrac{\pi}{4}$.
            \item Vế phải: $\arctan\left(\dfrac{x}{y}\right) = \arctan\left(\dfrac{\sqrt{\pi}/2}{\sqrt{\pi}/2}\right) = \arctan(1) = \dfrac{\pi}{4}$.
        \end{itemize}
        Vì hai vế bằng nhau, điểm này nằm trên đường cong.

        Để tính $y'(\frac{\sqrt{\pi}}{2})$ ta dùng phương pháp đạo hàm hàm ẩn. Lấy đạo hàm theo $x$ cả hai vế của phương trình đường cong, nhớ rằng $y$ là hàm của $x$, ta được:
        \[ \deriv{}{x}(xy) = \deriv{}{x}\left(\arctan\dfrac{x}{y}\right) \]
        \[ 1 \cdot y + x \cdot y' = \dfrac{1}{1 + (\frac{x}{y})^2} \cdot \deriv{}{x}\left(\dfrac{x}{y}\right) \]
        \[ y + xy' = \dfrac{1}{1 + \frac{x^2}{y^2}} \cdot \dfrac{1 \cdot y - x \cdot y'}{y^2} = \dfrac{y^2}{y^2+x^2} \cdot \dfrac{y - xy'}{y^2} = \dfrac{y-xy'}{x^2+y^2}. \]
        Bây giờ ta giải tìm $y'$:
        \[ (y + xy')(x^2+y^2) = y - xy' \]
        \[ yx^2 + y^3 + x^3y' + xy^2y' = y - xy' \]
        \[ x^3y' + xy^2y' + xy' = y - yx^2 - y^3 \]
        \[ y'(x^3 + xy^2 + x) = y(1 - x^2 - y^2) \]
        \[ y' = \dfrac{y(1-x^2-y^2)}{x(x^2+y^2+1)}. \]
        Tại $x=\frac{\sqrt{\pi}}{2}, y=\frac{\sqrt{\pi}}{2}$ ta có $x^2 = \frac{\pi}{4}, y^2 = \frac{\pi}{4}$. Thay vào, ta được:
        \[ y'\left(\frac{\sqrt{\pi}}{2}\right) = \dfrac{\frac{\sqrt{\pi}}{2}(1-\frac{\pi}{4}-\frac{\pi}{4})}{\frac{\sqrt{\pi}}{2}(1+\frac{\pi}{4}+\frac{\pi}{4})} = \dfrac{1-\frac{\pi}{2}}{1+\frac{\pi}{2}} = \dfrac{2-\pi}{2+\pi}. \]
        Phương trình tiếp tuyến với đường cong tại điểm $(\frac{\sqrt{\pi}}{2}, \frac{\sqrt{\pi}}{2})$ được cho bởi:
        \[ y - \dfrac{\sqrt{\pi}}{2} = y'\left(\dfrac{\sqrt{\pi}}{2}\right) \left(x - \dfrac{\sqrt{\pi}}{2}\right) = \dfrac{2-\pi}{2+\pi}\left(x-\dfrac{\sqrt{\pi}}{2}\right). \]
    \end{solution}
\end{example}

\subsection{Đạo hàm bậc cao}

Khi ta tính đạo hàm của một hàm số $f$, kết quả thu được, $f'$, cũng là một hàm số. Điều này mở ra một câu hỏi tự nhiên: Liệu ta có thể tiếp tục lấy đạo hàm của hàm số $f'$ không? Nếu $f'$ cũng là một hàm khả vi, đạo hàm của nó được gọi là \textbf{đạo hàm cấp hai} của $f$, và được ký hiệu là $f''$.
\[ f''(x) = (f'(x))' = \deriv{}{x} \left( \deriv{f}{x} \right). \]

\begin{example}
    Xét hàm số $f(x) = x^4 - 5x^2 + 2x$.
    \begin{itemize}
        \item Đạo hàm cấp một là: $f'(x) = 4x^3 - 10x + 2$.
        \item Đạo hàm cấp hai là: $f''(x) = (4x^3 - 10x + 2)' = 12x^2 - 10$.
    \end{itemize}
\end{example}

Về mặt vật lý, nếu $s(t)$ là hàm vị trí của một vật thể, thì đạo hàm cấp một $s'(t)$ biểu diễn vận tốc $v(t)$. Đạo hàm cấp hai $s''(t)$ chính là đạo hàm của vận tốc, biểu diễn cho \textbf{gia tốc} $a(t)$ của vật thể.
\[ a(t) = v'(t) = s''(t). \]
Gia tốc cho ta biết tốc độ thay đổi của vận tốc. Nếu gia tốc dương, vật thể đang tăng tốc. Nếu gia tốc âm, vật thể đang giảm tốc.

Ta có thể tiếp tục quá trình này. Đạo hàm của đạo hàm cấp hai là \textbf{đạo hàm cấp ba}, ký hiệu là $f'''$. Tổng quát, đạo hàm cấp $n$ của $f$, ký hiệu là $f^{(n)}$, được định nghĩa là đạo hàm của đạo hàm cấp $(n-1)$:
\[ f^{(n)}(x) = (f^{(n-1)}(x))'. \]

\begin{example}
    Tính các đạo hàm cấp cao của hàm số $f(x) = \cos x$.
    \begin{solution}
        Ta có:
        \begin{itemize}
            \item $f'(x) = -\sin x$
            \item $f''(x) = -\cos x$
            \item $f^{(3)}(x) = \sin x$
            \item $f^{(4)}(x) = \cos x$
        \end{itemize}
        Sau 4 lần lấy đạo hàm, ta quay trở lại hàm ban đầu. Chu kỳ này lặp lại vô hạn. Ta có thể viết công thức tổng quát cho đạo hàm cấp $n$ dựa trên phép chia lấy dư của $n$ cho 4:
        \[
        f^{(n)}(x) =
        \begin{cases}
            \begin{array}{rl}
                \cos x  & \text{nếu } n \equiv 0 \pmod{4} \\
                -\sin x & \text{nếu } n \equiv 1 \pmod{4} \\
                -\cos x & \text{nếu } n \equiv 2 \pmod{4} \\
                \sin x  & \text{nếu } n \equiv 3 \pmod{4}
            \end{array}
        \end{cases}
        \]

    \end{solution}
\end{example}

\begin{importantbox}
    Không phải mọi hàm số đều có đạo hàm ở mọi cấp. Một hàm số có thể khả vi một lần, nhưng đạo hàm cấp một của nó lại không khả vi.
\end{importantbox}

\begin{example}
    Xét hàm số $f(x) = x|x|$. Hàm số này có thể viết lại dưới dạng:
    \[ f(x) = \begin{cases}
        x^2 & \text{nếu } x \ge 0 \\
        -x^2 & \text{nếu } x < 0
    \end{cases} \]
    Ta tính đạo hàm cấp một. Với $x > 0$, $f'(x) = 2x$. Với $x < 0$, $f'(x) = -2x$. Tại $x=0$, ta có thể dùng định nghĩa để kiểm tra và thấy $f'(0)=0$. Do đó:
    \[ f'(x) = \begin{cases}
        2x & \text{nếu } x \ge 0 \\
        -2x & \text{nếu } x < 0
    \end{cases} = 2|x|. \]
    Hàm $f'(x) = 2|x|$ liên tục trên toàn bộ $\R$. Tuy nhiên, như ta đã biết, hàm trị tuyệt đối không khả vi tại $x=0$. Do đó, đạo hàm cấp hai $f''(0)$ không tồn tại. Hàm $f(x)$ chỉ khả vi đến cấp một tại $x=0$.
\end{example}

\subsection{Bài tập}

% [Quy tắc cơ bản: Tổng, Hiệu, Tích, Thương]
\begin{exercise}
    Tìm đạo hàm của các hàm số sau:
    \begin{enumerate}[label=(\alph*)]
        \item $f(x) = 6x^4 - 3x^2 + 9x^{2/3} - \sqrt{x}$
        \item $g(x) = (x^2 + \sqrt{x})(4x^3 - 3\sqrt[3]{x})$
        \item $h(t) = \dfrac{t^2+1}{t^2-t+1}$
    \end{enumerate}
\end{exercise}

% [Đạo hàm hàm lượng giác]
\begin{exercise}
    Tìm đạo hàm của các hàm số sau:
    \begin{enumerate}[label=(\alph*)]
        \item $y = \sqrt{x}\tan x$
        \item $y = \dfrac{\cos x}{1-\sin x}$
        \item $y = x^2 \cot x - \dfrac{2}{\sin x}$
    \end{enumerate}
\end{exercise}

% [Quy tắc chuỗi cơ bản]
\begin{exercise}
    Tìm đạo hàm của các hàm số sau:
    \begin{enumerate}[label=(\alph*)]
        \item $f(x) = (x^4 + 3x^2 - 2)^{5}$
        \item $g(t) = \sqrt[3]{1 + \tan t}$
        \item $h(x) = \sin(a^3+x^3)$
    \end{enumerate}
\end{exercise}

% [Quy tắc chuỗi phức tạp]
\begin{exercise}
    Tìm đạo hàm của các hàm số sau:
    \begin{enumerate}[label=(\alph*)]
        \item $y = (x^2+1)^4 (\sin x)^3$
        \item $y = \sin(\sqrt{x^2+1})$
        \item $y = \cos^4(\sin^3 x)$
    \end{enumerate}
\end{exercise}

% [Quy tắc chuỗi với hàm mũ và logarit]
\begin{exercise}
    Tìm đạo hàm của các hàm số sau:
    \begin{enumerate}[label=(\alph*)]
        \item $y = \sqrt[2024]{\ln(2023+x^2)e^{2025x}}$
        \item $y = \ln(\ln(\ln x))$
        \item $y = e^{\sin(e^x)}$
    \end{enumerate}
\end{exercise}

% [Các dạng đạo hàm đặc biệt]
% TAG: MOBIUS
\begin{exercise}
    Tìm đạo hàm của các hàm số sau:
    \begin{enumerate}[label=(\alph*)]
        \item $y = x \sin\dfrac{1}{x}$
        \item $y = x^x$
        \item $y = x^{\sin x}$
        \item $y = e^{e^{e^x}}$
    \end{enumerate}
\end{exercise}

% [Phương trình tiếp tuyến]
\begin{exercise}
    Hãy tìm phương trình của tiếp tuyến với đồ thị của mỗi hàm số sau tại giá trị $x$ cho trước.
    \begin{enumerate}[label=(\alph*)]
        \item $f(x) = \sqrt{x}$, tại $x=4$.
        \item $g(x) = \dfrac{x}{x^2+2}$, tại $x=1$.
    \end{enumerate}
\end{exercise}

% [Đạo hàm của hàm hợp trừu tượng - Dạng 1]
\begin{exercise}
    Cho $f, g$ là các hàm khả vi.
    \begin{enumerate}[label=(\alph*)]
        \item Cho $h(x) = f(g(\sin x))$. Tìm $h'(0)$ biết $g(0)=1, g'(0)=2, f'(1)=3$.
        \item Cho $F(x) = f(x \cdot f(x^2))$. Tìm $F'(2)$ biết $f(4)=2, f'(4)=3$.
    \end{enumerate}
\end{exercise}

% [Đạo hàm của hàm hợp trừu tượng - Dạng 2]
\begin{exercise} Thực hiện các yêu cầu sau:
    \begin{enumerate}[label=(\alph*)]
        \item Cho $g(x) = f(x^2 f(x))$. Biết $f(1)=2, f(2)=4, f'(1)=3, f'(2)=-1$. Tính $g'(1)$.
        \item Cho $F(x) = f(x f(x f(x)))$, với $f(1)=2, f(2)=3, f'(1)=4, f'(2)=5, f'(3)=6$. Tìm $F'(1)$. % (Kinh điển) 
    \end{enumerate}
\end{exercise}

% [Chứng minh công thức đạo hàm hàm ngược] (Kinh điển)
\begin{exercise}
    Hãy rút ra các công thức đạo hàm cho các hàm lượng giác ngược sau:
    \begin{enumerate}[label=(\alph*)]
        \item $(\arccos x)' = -\dfrac{1}{\sqrt{1-x^2}},\quad x \in (-1, 1)$.
        \item $(\arctan x)' = \dfrac{1}{1+x^2},\quad x \in \R$.
    \end{enumerate}
\end{exercise}

% [Tính đạo hàm hàm ngược]
\begin{exercise} Thực hiện các yêu cầu sau:
    \begin{enumerate}[label=(\alph*)]
        \item Cho $g(x) = \dfrac{x^{5}}{x^{4}+1}$ và cho $h$ là hàm ngược của $g$. Tính $h'(1/2)$.
        \item Cho $u(x) = \sqrt{x^3+x^2+x+1}$ và cho $v$ là hàm ngược của $u$. Tính $v'(2)$.
    \end{enumerate}
\end{exercise}

% [Đạo hàm hàm ẩn]
\begin{exercise}
    Tìm $y'=\dfrac{dy}{dx}$ cho các phương trình sau:
    \begin{enumerate}[label=(\alph*)]
        \item $x^2 + 2xy - y^2 + x = 2$
        \item $\sin(xy) = x^2 - y$
        \item $e^y \cos x = 1 + \sin(xy)$
    \end{enumerate}
\end{exercise}

% [Tiếp tuyến của đường cong cho bởi hàm ẩn]
\begin{exercise}
    Viết phương trình tiếp tuyến của đồ thị tại điểm được chỉ định.
    \begin{enumerate}[label=(\alph*)]
        \item $x^3+y^3=6xy$ tại điểm $(3,3)$. % (Kinh điển)
        \item $x^2+y^2=25$ tại điểm $(3,-4)$. % (Kinh điển)
        \item $y \sin(2x) = x \cos(2y)$ tại điểm $(\pi/2, \pi/4)$.
    \end{enumerate}
\end{exercise}

% [Tiếp tuyến của các đường cong đặc biệt]
\begin{exercise}
    Viết phương trình tiếp tuyến của các đường cong sau tại điểm cho trước.
    \begin{enumerate}[label=(\alph*)]
        \item (Cardioid) $x^2+y^2=(2x^2+2y^2-x)^2$ tại điểm $(0, 1/2)$.
        \item (Lemniscate) $2(x^2+y^2)^2 = 25(x^2-y^2)$ tại điểm $(3,1)$.
    \end{enumerate}
    Bonus: hãy thử dùng các phần mềm vẽ đồ thị (vd: Geogebra) để hiển thị 2 đường cong ở trên, hãy đoán xem hình dạng của 2 đồ thị đó là gì?
\end{exercise}

% [Bài toán tốc độ biến thiên liên quan - Hình học]
\begin{exercise} Dùng các công thức đạo hàm, trả lời các câu hỏi sau:
    \begin{enumerate}[label=(\alph*)]
        \item  Một cái thang dài 50 mét đang dựa vào tường. Khi đỉnh thang đang ở cách nền 30 mét thì thang bị trượt, đỉnh thang tuột xuống với vận tốc 3 mét mỗi giây. Hỏi đáy thang trượt xa khỏi bức tường với vận tốc bao nhiêu? %(Kinh điển) TAG:MOBIUS
        \item Một người đứng trên cầu cao 15m so với mặt nước, kéo một chiếc thuyền vào bờ bằng một sợi dây với tốc độ 1 m/s. Thuyền tiến vào bờ nhanh như thế nào khi nó còn cách bờ 8m?
    \end{enumerate}
\end{exercise}

% [Bài toán tốc độ biến thiên liên quan - Vật lý & Kinh tế]
\begin{exercise} Dùng các công thức đạo hàm, giải các bài toán sau:
    \begin{enumerate}[label=(\alph*)]
        \item Một bồn nước hình trụ đáy tròn bán kính 2 m được bơm nước vào với tốc độ 3 m$^3$/phút. Hỏi mực nước trong bồn đang dâng lên với tốc độ bao nhiêu?
        \item Chi phí $C$ (tính bằng triệu đồng) để sản xuất $x$ đơn vị sản phẩm được cho bởi $C(x) = 100 + 0.05x^2$. Nếu sản lượng đang tăng với tốc độ 10 đơn vị/ngày, chi phí đang tăng với tốc độ bao nhiêu khi sản lượng là 200 đơn vị?
    \end{enumerate}
\end{exercise}

% [Đạo hàm cấp cao]
\begin{exercise} Thực hiện các yêu cầu sau:
    \begin{enumerate}[label=(\alph*)]
        \item Tìm $y''$ nếu $x^4+y^4=16$.
        \item Hãy tính đạo hàm cấp 3 của hàm số $f(x) = \dfrac{\sin x}{e^x}$.
    \end{enumerate}
\end{exercise}

% [Tìm công thức đạo hàm cấp $n$]     (Kinh điển) 
\begin{exercise} 
    Hãy tìm đạo hàm cấp $n$ của mỗi hàm sau:
    \begin{enumerate}[label=(\alph*)]
        \item $f(x) = x^{k}$ (với $k$ là số nguyên, $k \ge n$)
        \item $f(x) = \dfrac{1}{ax+b}$
        \item $f(x) = \ln(1+x)$
        \item $f(x) = \cos(ax)$
    \end{enumerate}
\end{exercise}

% (Kinh điển) [Công thức Leibniz] 
\begin{exercise}
    Bằng phương pháp quy nạp, hãy chứng minh công thức Leibniz cho đạo hàm cấp $n$ của một tích: Nếu $f$ và $g$ có đạo hàm đến cấp $n$ thì
    \[ (f \cdot g)^{(n)} = \sum_{k=0}^{n} C_{n}^{k} f^{(k)} \cdot g^{(n-k)}. \]
    Áp dụng, tính đạo hàm cấp 2025 của hàm $h(x) = (x^2+1)e^{2x}$.
\end{exercise}

% [Ứng dụng công thức Leibniz]
\begin{exercise}
    Sử dụng công thức Leibniz, tìm $y^{(n)}$ với:
    \begin{enumerate}[label=(\alph*)]
        \item $y = x e^x$.
        \item $y = (1-x^2)\cos x$.
        \item $y = x^3 \ln x$.
    \end{enumerate}
\end{exercise}
